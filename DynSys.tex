%! TEX = pdflatex
% 
% // cSpell:ignore fontsize totalheight keepaspectratio docspec
\documentclass{tufte-handout}

%handling bibliography



%fancy math fonts
\usepackage{mathrsfs}

%fancy cases and steps
\usepackage{enumerate,mdwlist}
\usepackage{enumitem}



\usepackage{amsmath}  % extended mathematics


\newcommand\n[1]{\lVert#1\rVert}
\newcommand{\ok}{\mathcal{O}_{\K }}
\newcommand{\EL}{\mathbb{E}^1}
\newcommand{\EP}{\mathbb{E}^2}
\newcommand{\EN}{\mathbb{E}^n}
\newcommand{\afn}{\mathbb{A}}%affine
\newcommand{\lcs}{\mathcal{Z}}%locus
\newcommand{\rdl}{\mathcal{I}}%radical ideal
\newcommand{\mf}[1]{\mathfrak{#1}}
\newcommand{\spt}{\supseteq}
\DeclareMathOperator{\res}{res}
\newcommand{\bsts}[2]
{%big subset
\left\{#1 \middle| 
\begin{gathered}
#2 
\end{gathered} 
\right\}
}
\newcommand{\basis}[8]{% v m v b1 b2
\begin{center}
	\begin{tikzcd} 
	#1  \ar[r,"#2 (\cdot )"]& #3 \\
	#6 \ar[u,"#4 "] 
	\ar[r,"#8 "]
	 & #7   \ar[u,"#5 "].
	\end{tikzcd}
\end{center}
}
\DeclareMathOperator{\Nm}{N}
\DeclareMathOperator{\coker}{coker}
\DeclareMathOperator{\im}{Im}
\DeclareMathOperator{\coim}{coim}
\DeclareMathOperator{\Spec}{Spec}
\DeclareMathOperator{\mSpec}{mSpec}
%%%%%%%%%%%%%Complex Analysis Package%%%%%%%%%%%%%
\newcommand{\D}{\mathbb{D}}%the unit disc

%%%%%%%%%%%%%Complex Analysis Package%%%%%%%%%%%%%

%%%%%%%%%%%%%Algebra Package%%%%%%%%%%%%%
\newcommand{\IP}[2]{\langle #1,#2 \rangle}
\newcommand{\veb}{\mathbf{B}}
\DeclareMathOperator{\End}{End}
\DeclareMathOperator{\Aut}{Aut}\newcommand{\sg}{\leq}
\newcommand{\nsg}{\unlhd}
%%%%%%%%%%%%%Algebra Package%%%%%%%%%%%%%


%%%%%%%%%%%%%Measure Package%%%%%%%%%%%%%
\newcommand{\MM}{\mathfrak{M}}
\newcommand{\LL}{\mathcal{L}}
\newcommand{\CC}{\mathcal{C}}
%%%%%%%%%%%%%Measure Package%%%%%%%%%%%%%
\newcommand{\case}{[label=\textbf{Case \arabic*:}]}
\newcommand{\stp}{[label=\textbf{Case \arabic*:}]}

%%%%%%%%%%%%%Topology Package%%%%%%%%%%%%%
\newcommand{\aaa}{\mathscr{A}}
\newcommand{\TT}{\mathscr{T}}
\newcommand{\PP}{\mathscr{P}}
\newcommand{\BB}{\mathscr{B}}
\newcommand{\FF}{\mathscr{F}}
\newcommand{\UU}{\mathscr{U}}
\newcommand{\VV}{\mathscr{V}}
\newcommand{\WW}{\mathscr{W}}
\newcommand{\sss}{\mathcal{S}}
\newcommand{\sbt}{\subseteq}%subset
\newcommand{\sts}[2]{\left\{#1\middle|#2\right\}}
\newcommand{\unn}[3]{\underset{#1 \in #2}{\bigcup}#3} %union
\newcommand{\inn}[3]{\underset{#1 \in #2}{\bigcap}#3} %intersection
%\newcommand{\ar}[2]{[#1,"#2"]}
\newcommand{\apn}[1]{\ar[#1,"open ",symbol=\sbteq]}%arrow open subset
%%%%%%%%%%%%%Topology Package%%%%%%%%%%%%%


%%%%%%%%%%%%%Manifold Package%%%%%%%%%%%%%
\newcommand{\kk}{$C^k $}
\newcommand{\cy}{$C^\infty $}
\newcommand{\ck}{$C^k $}
\newcommand{\pl}[2]{\frac{\partial #1}{\partial #2}}%partial derivative
\newcommand{\nti}[4]{\int_{#1}^{#2}#3\,\mathrm{d}#4}%integration
\newcommand{\bv}[2]{(\frac{\partial }{\partial x^{#1 }} )^{#2 }}% basis vector field w.r.t a coordinate
\newcommand{\cv}[1]{(\frac{\partial }{\partial t} )^{#1 }}% curve vector field 
\newcommand{\dbv}[2]{(\mathrm{d}x^{#1 } )_{#2 }}% dual basis vector field w.r.t a coordinate
\newcommand{\dfv}[1]{(\mathrm{d}f )_{#1 }}% Dual Vector field of a Function w.r.t a coordinate
%%%%%%%%%%%%%Manifold Package%%%%%%%%%%%%%


%%%%%%%%%%%%%Miscellaneous%%%%%%%%%%%%%
\DeclareMathOperator{\diam}{diam}
\newcommand{\C}{\mathbb{C}}
\newcommand{\K}{\mathbb{K}}
\newcommand{\R}{\mathbb{R}}
\newcommand{\Z}{\mathbb{Z}}
\newcommand{\Q}{\mathbb{Q}}
\newcommand{\N}{\mathbb{N}}
\newcommand{\uln}[1]{\underline{#1}}%underline
\newcommand{\topic}[1]{\noindent{\textbf{#1}}}
\newcommand{\ud}[1]{\underline{#1}}%underline
\newcommand{\df}[1]{\overset{def}{#1}}%Definition
\newcommand{\cl}[1]{\overline{#1}}%closed 
\newcommand{\tcr}[1]{\textcolor{red}{#1} }%textcolor red
\newcommand{\xra}[3]{#1 \xrightarrow{#2} #3}%xrightarrow
\newcommand{\ed}[2]{|#1-#2|}%Euclidean Distance
\newcommand{\sqc}[3]{#1 \rightarrow #2, \ as \  #3 \rightarrow \infty}%sequence converges
\newcommand{\bij}{\longleftrightarrow}
\newcommand{\ctd}{$\longrightarrow \longleftarrow $}
\newcommand{\bc}[2]{\genfrac{(}{)}{0pt}{}{#1}{#2}}
\newcommand{\ovs}[2]{\overset{#1}{#2}}%overset
\newcommand{\ssmm}[2]{\sum\limits_{\substack{#1}}^{#2}}%sum
\newcommand{\pprr}[2]{\prod\limits_{#1}^{#2}}%product
\newcommand{\uunn}[2]{\bigcup_{#1}^{#2}}%union
\newcommand{\iinn}[2]{\bigcap_{#1}^{#2}}%intersection



\usepackage{import}
\usepackage{xifthen}
\usepackage{pdfpages}
\usepackage{transparent}

\newcommand{\incfig}[1]{%
    \def\svgwidth{\columnwidth}
    \import{\string~/ObsidianOne/figures/}{#1.pdf_tex}
}
\usepackage{quiver}
%\geometry{showframe} % display margins for debugging page layout
\usepackage{graphicx} % allow embedded images
  \setkeys{Gin}{width=\linewidth,totalheight=\textheight,keepaspectratio}
\usepackage{svg}
\usepackage{transparent}
\usepackage{booktabs} % book-quality tables
\usepackage{units}    % non-stacked fractions and better unit spacing
\usepackage{multicol} % multiple column layout facilities
\usepackage{lipsum}   % filler text
\usepackage{fancyvrb} % extended verbatim environments
  \fvset{fontsize=\normalsize}% default font size for fancy-verbatim environments

% Standardize command font styles and environments
\newcommand{\doccmd}[1]{\texttt{\textbackslash#1}}% command name -- adds backslash automatically
\newcommand{\docopt}[1]{\ensuremath{\langle}\textrm{\textit{#1}}\ensuremath{\rangle}}% optional command argument
\newcommand{\docarg}[1]{\textrm{\textit{#1}}}% (required) command argument
\newcommand{\docenv}[1]{\textsf{#1}}% environment name
\newcommand{\docpkg}[1]{\texttt{#1}}% package name
\newcommand{\doccls}[1]{\texttt{#1}}% document class name
\newcommand{\docclsopt}[1]{\texttt{#1}}% document class option name
\newenvironment{docspec}{\begin{quote}\noindent}{\end{quote}}% command specification environment
%%%%%%%%%%%%%%%%%%%%%%%%%%%%%%%%%%%%%%%%%%%%%%%%%%%%%%%%%%%%%%%%%%%%%%%%%%%%%%%%%%%%%%%%%%%%%%%%%%%%%%
% add numbers to chapters, sections, subsections
\setcounter{secnumdepth}{2}
\usepackage{xcolor}
\definecolor{g1}{HTML}{077358}
\definecolor{g2}{HTML}{00b096}
% chapter format  %(if you use tufte-book class)
%\titleformat{\chapter}%
%{\huge\rmfamily\itshape\color{red}}% format applied to label+text
%{\llap{\colorbox{red}{\parbox{1.5cm}{\hfill\itshape\huge\color{white}\thechapter}}}}% label
%{2pt}% horizontal separation between label and title body
%{}% before the title body
%[]% after the title body

% section format
\titleformat{\section}%
{\normalfont\Large\itshape\color{g1}}% format applied to label+text
{\llap{\colorbox{g1}{\parbox{1.5cm}{\hfill\color{white}\thesection}}}}% label
{1em}% horizontal separation between label and title body
{}% before the title body
[]% after the title body

% subsection format
\titleformat{\subsection}%
{\normalfont\large\itshape\color{g2}}% format applied to label+text
{\llap{\colorbox{g2}{\parbox{1.5cm}{\hfill\color{white}\thesubsection}}}}% label
{1em}% horizontal separation between label and title body
{}% before the title body
[]% after the title body

%%%%%%%%%%%%%%%%%%%%%%%%%%%%%%%%%%%%%%%%%%%%%%%%%%%%%%%%%%%%%%%%%%%%%%%%%%%%%%%%%%%%%%%%%%%%%%%%%%%%%%
\usepackage{color-tufte}
%%%%%%%%%%%%%%%%%%%%%%%%%%%%%%%%%%%%%%%%%%%%%%%%%%%%%%%%%%%%%%%%%%%%%%%%%%%%%%%%%%%%%%%%%%%%%%%%


\setcounter{tocdepth}{4}
\title{Dynamical System}
\begin{document}
%稳定性理论部分思维导图
%要求:
% DONE 自治系统李雅普诺夫第一方法(一阶近似法)
% DONE 李雅普诺夫第二方法(构造V函数)的全部定理内容、每个定理列举至少一道习题。
% TODO 非自治系统相关内容,如线性非自治方程的稳定性、非自治方程李雅普诺夫函数的相关结论。
% TODO 李雅普诺夫函数构造的相关问题。
% DONE 书中58页定理3.4、77页定理3.11等定理的相关证明。
\maketitle
\tableofcontents
\section{Stability}
\begin{marginfigure}
    \centering
        \incfig{stability}
	\caption{Stable}
\end{marginfigure}

\begin{marginfigure}
    \centering
		\incfig{attractive}
	\caption{Attractive}
\end{marginfigure}
\begin{definition} 

\begin{itemize}
	\item \textbf{Stability:} the zero solution is called \emph{stable} if 
\[ \lim_{x_0\to 0}\sup_{t > t_0}x(t,t_0,x_0)=0 \]
        \item \textbf{Attractivity} the zero solution is called \emph{attractive} if

		[ \forall x_0 \text{in defined region,} \lim_{t\to\infty}x(t,t_0,x_0)=0 \]
	\item \textbf{Asymptotical Stability:} the zero solution is called \emph{asymptotically stable} if it is both stable and attractive.
	\item \textbf{Unstability:} A fixed point $x^*$ is said to be unstable if it is not stable.
\end{itemize}
\end{definition}
\subsection{Autonomous Systems}
\paragraph{Linear Systems}

The one parameter group is given by the matrix exponential \[ x(t) = e^{(t-t_0)A}x_0, \]
where $A$ is the matrix of the linear system $\partial_t x = Ax$.

The zero solution of the system is stable $\iff$ \[ \lim_{x_0 \to \infty} (\sup{t>t_0} e^{(t-t_0)A}) x_0 =0 \]

For simplicity we deal with the linear system on $\mathbb{C}^n$. (In fact the real case could be reduced by complexification).
\begin{lemma} 
	For any $P \in GL_n(\mathbb{C})$, the zero solution of the new dynamical system for $y= Px \in \mathbb{C}^n$ given the vector field \[ \partial_t y = (PAP^{-1})y \] is stable(resp. attractive) if and only if the zero solution of the original system is stable(resp. attractive).
\end{lemma}
\begin{formalproof} 
The one parameter group of the new system is given by \[ y(t) = e^{(t-t_0)PAP^{-1}}y_0 = Pe^{(t-t_0)A}P^{-1}y_0, \]
hence  
\begin{align*}
   & \text{the zero solution in the new system is stable} \\
	\iff & P \lim_{y_0 \to 0} (\sup_{t>t_0} e^{(t-t_0)A}) P^{-1} y_0 = P  \lim_{P x_0 \to 0} (\sup_{t>t_0} e^{(t-t_0)A}) x_0 =0  \\
	\iff & ({\text{P linear bijection hence homeomorphism}})  \lim_{x_0 \to 0} (\sup_{t>t_0} e^{(t-t_0)A}) x_0 =0 \\
	\iff & \text{the zero solution in the original system is stable}
	\end{align*}
	
	
Similarly 
\begin{align*}
    & \text{the zero solution in the new system is attractive} \\
	\iff & \forall y_0 \text{we have} P \lim_{t \to \infty} ( e^{(t-t_0)A}) P^{-1} y_0 = P  \lim_{t \to \infty} (e^{(t-t_0)A}) x_0 =0  \\
	\iff & \forall x_0 \text{we have} \lim_{t \to \infty} ( e^{(t-t_0)A}) x_0 =0 \\
	\iff & \text{the zero solution in the original system is attractive}
\end{align*}
\end{formalproof}
\begin{lemma} Given  diagonal blocks matrix $A:={A_1,...,A_k}$
	with	the subsystems $x^i \in \C^{n_i}$ with vector fields defined by \[ \partial_t x^i = A_i x, \]

	then the zero solution of the system is stable(resp. attractive) if and only if the zero solution of \emph{each} subsystem is stable(resp. attractive).
\end{lemma}
\begin{formalproof} 
\begin{align*}
	& \text{the zero solution in the original system is stable} \\
	\iff & \lim_{x_0 \to 0} (\sup_{t>t_0} e^{(t-t_0)A}) x_0 =0 \\
\end{align*}
where \[ e^{(t-t_0)A}) x_0 =
\begin{pmatrix}
	e^{(t-t_0)A_1})& \cdots & 0 \\
	\vdots & \ddots & \vdots \\
	0 & \cdots & e^{(t-t_0)A_k}) 
\end{pmatrix}
\begin{pmatrix}
	x_0^1 \\
	\vdots \\
	x_0^k
\end{pmatrix}
= 
\begin{pmatrix}
	e^{(t-t_0)A_1}) x_0^1 \\ 
	\vdots \\
	e^{(t-t_0)A_k}) x_0^k
\end{pmatrix}
 \]
 
 Hence 
 \begin{align*}
 & 	 \lim_{x_0 \to 0} (\sup_{t>t_0} e^{(t-t_0)A}) x_0 =0 \\
 	 \iff & \lim_{x_0 \to 0} (\sup_{t>t_0} e^{(t-t_0)A_j}) x_0^j =0  \forall 1 \leq j \leq k \\
 	 \iff & \text{the zero solutions of all subsystems are stable.}
 \end{align*}
 
 Similarly
 \begin{align*}
 	&\text{the zero solution in the original system is attractive} \\
	\iff & \lim_{t\to \infty} e^{(t-t_0)A}) x_0 =0  \forall x_0 \\
		\iff & \lim_{t\to \infty} e^{(t-t_0)A_j}) x_0^j =0  \forall x_0,1 \leq j \leq k \\
	\iff & 	\text{the zero solutions in all subsystems are attractive}
 \end{align*}
 \end{formalproof}
\begin{lemma} 
For a "Jordan" system defined by \[ \partial_t x = 
\begin{bmatrix}
	\lambda & 1 & 0 & \cdots & 0 \\
	0 & \lambda & 1 & \cdots & 0 \\
	\vdots & \vdots & \vdots & \ddots & \vdots \\
	0 & 0 & 0 & \cdots & \lambda
\end{bmatrix}
x =: J_m(\lambda)x, 
\]
the zero solution is stable if and only if $\Re(\lambda)<0$ or ($\Re(\lambda)=0$ and $m=1$).

the zero solution is attractive if and only if $\Re(\lambda)<0$.
\end{lemma}
\begin{formalproof} 
For the euclidean topology over $M_m(\mathbb{R})$ we use the standard norm \[ \n{A}:= \sqrt{\sum_{ij} a^2_{ij}} \]

We use the fact that in phase space the limit is the origin $\iff$ the norm is tending to 0, without explicitly ramble about it.

A quick computation: 
\[ e^{(t-t_0)J_m(\lambda)} = e^{(t-t_0)\lambda}
\begin{pmatrix}
	1 & \frac{t-t_0}{1!} & \cdots & \frac{(t-t_0)^{m-1}}{(m-1)!} \\
	0 & 1 & \cdots & \frac{(t-t_0)^{m-2}}{(m-2)!} \\
	\vdots & \vdots & \ddots & \vdots \\ 
	0 & 0 & \cdots & \frac{t-t_0}{1!} \\
	0 & 0 & \cdots & 1
\end{pmatrix}
 \]
 hence 
\begin{align*}
	n(t)&:=\n{e^{(t-t_0)J_m(\lambda)}} \\
	    &= e^{(t-t_0)\Re{\lambda}}\sqrt{m \cdot 1^2+ (m-1)(\frac{t-t_0}{1!})^2 + \cdots + 1 \cdot (\frac{(t-t_0)^{m-1}}{(m-1)!}})^2 \\
	    &=: e^{(t-t_0)\Re{\lambda}} \sqrt{P}(t)
\end{align*}
\text{It suffices to consider the following cases} \\
\begin{enumerate}[label=\textbf{Case \Roman*:},leftmargin=1.3cm]
	\item For $\Re(\lambda)<0$ we need to show the zero solution is stable and attractive. 
 Firstly, \[ \n{e^{(t-t_0)J_m(\lambda)} x_0} \leq \n{e^{(t-t_0)J_m(\lambda)}} \cdot \n{x_0}, \]
 where $n(t)=\n{e^{(t-t_0)J_m(\lambda)}} \to 0 $ as $t \to \infty$, hence the zero solution is attractive.
 
 Secondly, we could do "one point compactification" to get the space \[ [t_0,\infty)\cup {\infty}, \]
 and extend $n$ by mapping $\infty \mapsto 0$. And since $n(t)\to 0$, the extended mapping is a continuous real valued function over a compact space, hence admits a maximum, say $M$. Therefore 
 \[ \n{e^{(t-t_0)J_m(\lambda)} x_0} \leq \n{e^{(t-t_0)J_m(\lambda)}} \cdot \n{x_0} \leq M \cdot \n{x_0}, \]
 hence the zero solution is stable.
	\item For ($\Re(\lambda)=0$ and $m=1$) we need to show the zero solution is stable. But $n(t)\equiv 1$, hence  \[ \n{e^{(t-t_0)J_m(\lambda)} x_0} =  \n{x_0}, \]
thus the zero solution is stable.

	\item For $\Re(\lambda)>0$ we need to show the zero solution is neither stable nor attractive.
	Given $x_0 \neq 0$, without loss of generality, let $(x_0)_1 =1 $(since we could focus our attention at a small ball centered at origin, which is homeomorphic to $\mathbb{E}^m$)
	
	therefore 
	\begin{align*}
		\n{e^{(t-t_0)J_m(\lambda)} x_0} &\geq n{(e^{(t-t_0)J_m(\lambda)})_1}  \\
		& = e^{(t-t_0)\Re{\lambda}} \to \infty,
	\end{align*}
	it is unbounded, hence the limit as $t\to \infty$ could not exist, hence the zero solution is NOT attractive.
	
	Now consider the sequence of initial positions ${(x_0)_n} :=\{ \frac{1}{t'-t_0} \}e_1 $ where $t'$ takes $t_0+n$. Of course the sequence is tending to the origin. But as $t_0+n$ grows, \[ \n{e^{(t-t_0)J_m(\lambda)} x_0} = \frac{e^{n\Re{\lambda}}}{n} \to \infty, \]
	
	hence the zero solution is NOT stable.
	
	\item For $\Re(\lambda)=0$ we need to show the zero solution is NOT attractive. WLG, we consider the initial position $x_0=e_1$(since we could focus our attention at a small ball centered at origin, which is homeomorphic to $\mathbb{E}^m$)
	\[ \n{e^{(t-t_0)J_m(\lambda)} x_0} =  \n{e_1}=1, \]
	which is independent of $t$, hence the zero solution is NOT attractive.
	\item For ($\Re(\lambda)=0$ and $m>1$)we need to show the zero solution is NOT stable.

	WLG, we consider the path $(x_0)_\delta = \delta e_2$. hence  \[ \n{e^{(t-t_0)J_m(\lambda)} x_0} =  \delta (t-t_0 + 1), \]
    now consider the sequence $(x_0)_{n}= \frac{1}{(n+t_0)} e_2$ hence the norm is just \[  \frac{t-t_0+1}{n+t_0} \to 1, \]
    hence as the sequence of pts in phase space approaching the origin, the norm is no where near touching the 0, i.e., the zero solution could not possibly be stable.
\end{enumerate}

 \end{formalproof}
\begin{theorem} 
	\begin{enumerate}
		\item 
	the zero solution of the linear system $\partial_t x = Ax$ is assymtotically stable  \\ $\iff$ All the subsystems with vector fields $J_{n_i}(\lambda_i)$ is both attractive and stable \\ $\iff$ All the eigenvalues of $A$ have negative real parts.
\item
	the zero solution of the linear system $\partial_t x = Ax$ is stable \\ $\iff$ All the subsystems are stable \\ $\iff$ All the eigenvalues of $A$ have non-positive real parts, and for $\lambda_i=0$ the jordan block is of size $n_i=1$.

	In particular, if there exists $\lambda_i>0$ or ($\lambda_i=0$ and $n_i>1$), then the zero solution is unstable.
\end{enumerate}
\end{theorem}
\begin{formalproof} it follows directly from the lemmas above. \end{formalproof}

\paragraph{Liapunov2: V Function}
\underline{Intuition}
If using an Euclidean metric(inner product), we find \[ r(f(x),x)\leq 0  \text{'} \iff \text{'}  \nabla r^2(x) \cdot f(x) \leq 0 \]
i.e., the vector field is pointing inwards, then given $x_0$, we could have a pretty decent predection of $x$ having smaller magnitude as time evolves.


In fact, we could consider a broader class of positive($\geq0$) definite($=0 \iff x=0$) functions(Liapunov V functions) in a neighborhood of $x=0$ rather than just the global norm $r^2$, which are not necessarily quadratic, but still retain the sense that $f(x)$ is "pointing inwards"($\nabla V(x) \cdot f(x) \leq 0$).
\begin{theorem} 
	If such a function $V(x)$ exists, then the system is stable.(Later we will see the existence of $V$ is not only sufficient, but also necessary )

	Moreover, if the set of points where $\nabla V(x)$ is orthogonal to $f(x)$ does not contain any other phase curves of the system(in particular, if $\nabla V(x) \cdot f(x) $ is positive definite), then the system is asymptotically stable.

\end{theorem} 

Note that all problems come from the textbook \cite{2001} Problem set 3, Page 122-124.

\underline{Problem 14(1)} Let \[ V(x)=\frac{x_1^2+x_2^2}{2}, \]
then \[ \nabla V (x) \cdot 
\begin{pmatrix}
	-x_1+x_1x^2 \\
	-2x_1^2 x_2 -x_2
\end{pmatrix}
= 
-x_1^2-x_2^2-(x_1x_2)^2 \leq 0.
\]

Using the same intuition of the direction of $f(x)$ we would have 2 sufficient conditions of the unstability of the system.
\begin{theorem} 
If there exists a function $U(x)$ in a small neighborhood of $x=0$ such that
\begin{enumerate}

	\item $U(0)=0$ and at some pt. $\overline{x_0}$ in the small nbh. we have $U(\overline{x_0}) > 0$,
\item $\nabla U(x) \cdot f(x) > 0$ for all $x$ in the nbh. except $x=0$(i.e., positive definite; in other words, the vector field is pointing outwards),

then the system is unstable.
\end{enumerate}
\end{theorem}
\underline{Problem 2(4)} Let \[ U(x)=\frac{x_1^2+x_2^2}{2}, \]
which is postive definite,
then \[ \nabla U(x) \cdot
\begin{pmatrix}
	x_1-x_2+x_1x_2+x_1^3 \\
	x_1+x_2-x_1^3-x_2^2
	\end{pmatrix}
	= r^2 + O(r^3) = r^2(1+ O(r))
\]
where $r=\sqrt{x_1^2+x_2^2}.$ Hence there exists $\delta> 0$ s.t. for all $(x_1,x_2)\in B(O,\delta)/{O}$ we have \[ \nabla U(x) \cdot f(x_1x_2) >0. \]
therefore the zero solution is unstable.
\begin{theorem} 
If there exists a function $U(x)$ in a small neighborhood of $x=0$ such that
\begin{enumerate}
    \item $U(0)=0$ and for any small nbh. of $x=0$ there exists a pt. $\overline{x_0}$  s.t. $U(\overline{x_0}) > 0$,
    \item \[ \nabla U(x) \cdot f(x) = \lambda U(x)+ W(x), \]
where $\lambda>0, W(x) \geq 0$(essentially the same as $\partial_t (U(x(t))) \geq \lambda U(x)$),
\end{enumerate}

then the system is unstable.
\end{theorem}
\begin{formalproof} 
	(Proof by contradiction)
	Assume that the system is stable, then for any small $\epsilon>0$ we could find $\delta>0$ s.t. whenever \[ x_0^T x_0 < \delta, \]
	for any time $t$ after $t_0$ we could control \[ x(t,t_0,x_0)^Tx(t,t_0,x_0)< \epsilon.  \]

For such a small nbh. of $B(\delta,0)$ we could find $\overline{x_0} \in B(\delta,0)$ s.t. \[ U(\overline{x_0})> 0 \]

Now since $\partial_t (U(x(t,t_0,\overline{x_0}))) \geq \lambda U(x(t,t_0,\overline{x_0}))$, by Gronwall's Inequality, \[ \partial_t U(x(t,t_0,\overline{x_0})) \geq \lambda U(x(t,t_0,\overline{x_0})) \geq \lambda U(\overline{x_0}) e^{\lambda(t-t_0)}> 0. \]

Since $U(x)$ is continuous and $U(0)=0$ there exists $\epsilon >\eta> 0$ s.t. for all time $t>t_0$ \[ x(t,t_0,\overline{x_0})^T x(t,t_0,\overline{x_0}) \geq \eta, \]

therefore, 
\begin{align*}
  U(x(t,t_0,\overline{x_0})) &= U(\overline{x_0})+ \int_{t_0}^t \partial_t U(x(\tau,t_0,\overline{x_0})) \text{d}\tau \\
                             &\geq U(\overline{x_0})+ \int_{t_0}^t \lambda U(x(\tau,t_0,\overline{x_0})) + W(x(\tau,t_0,\overline{x_0})) \text{d}\tau
\end{align*}
but the function inside the integral is actually $\geq \lambda U(\overline{x_0})$, hence \[ U(x(t,t_0,\overline{x_0})) \geq U(\overline{x_0}) + \lambda U(\overline{x_0})(t-t_0) \to \infty, \]

as time evolves $t\to\infty$. 

However, the integral curve $x(t,t_0,\overline{x_0})$ is bounded inside $B(0,\epsilon)$, and since $U(x)$ is continuous, the values of $U(x(t,t_0,\overline{x_0}))$ is also bounded, leading to the contradiction, completing the proof.
\end{formalproof}

\underline{Problem 2(7)} Let \[ U(x)=\frac{x_1^2-x_2^2}{2}, \]
then \[ \nabla U(x) \cdot 
	\begin{pmatrix}
		x_1+x_2+x_1x_2^2 \\
		2x_1+x_2-x_1^2 x_2
	\end{pmatrix}
	= x_1^2-x_2^2+2(x_1x_2)^2=2U+W,
\]
where $W\geq 0$, therefore the zero solution of the system is stable.
\paragraph{Liapunov1: Linear Appoximation}
\begin{theorem} 
Let $f$ be a non-autonomous vector field on the phase space satisfying:
\begin{enumerate}
	\item there exists $k\geq 0$ s.t. \[ \n{f(t,x_1)-f(t,x_2)} \leq k \n{x_1-x_2}, \]
	\item $f$ is of order greater than $\n{x}$. more precisely, \[ \lim_{x\to 0} \sup_{t>t_0} \frac{\n{f(t,x)}}{\n{x}}=0. \]
\end{enumerate}

Suppose $A$ has no pure imaginary(0 included) eigenvalue, then
\begin{align*}
& \text{the zero solution of the vector field } Ax+f \text{ is asymptotically stable}\\
\iff & \text{the zero solution of the vector field } Ax \text{ is asymptotically stable} \\
\iff & \text{All eigenvalues of } A \text{ have negative real part}.
\end{align*}

Likewise,
\begin{align*}
& \text{the zero solution of the vector field } Ax+f \text{ is unstable}\\
\iff & \text{the zero solution of the vector field } Ax \text{ is unstable} \\
\iff & \text{At least one eigenvalue of } A \text{ has positive real part}.
\end{align*}
 \end{theorem}
\begin{sketchproof} 
%ingredient: [label=\textbf{Case \Roman*:},leftmargin=1.3cm]
	It suffices to show \\
\begin{enumerate}[label=\textbf{Case \Roman*:}]
	\item 
		\begin{align*} & \text{All eigenvalues of } A \text{ have negative real part}.\\ \implies & \text{the zero solution of the vector field } Ax+f \text{ is asymptotically stable}
 \end{align*}
\begin{enumerate}[label=\textbf{Step \Roman*:},leftmargin=0.5cm]
	\item Constructing a Liapunov function $V(x)$
\begin{enumerate}[label=\textbf{Method \Roman*:},leftmargin=1.0cm]
	\item using $(A^T * + * A)^{-1}(-E)$ as the(note that all eigenvlues of $A$ lies in the left half plane hence could not be symmetric about the 0 on the complex plane thus the mapping is a linear isomorphism by \ref{lemma:isolinear}) quadratic form for $V$.
	\item finding a (almost) proper basis(to controll $\nabla V \cdot A$), and use the standard norm for $V$.
\end{enumerate}
\item (cf. \cite{arnold1992ordinary}) Controling $\nabla V(x) \cdot f(x)$ in a small enough nbh. of the origin of the phase space.
\end{enumerate}
\item 
			\begin{align*} & \text{At least one eigenvalue of } A(\text{ say } \lambda) \text{ has positive real part}.\\ \implies & \text{the zero solution of the vector field } Ax+f \text{ is unstable}
 \end{align*}
\begin{enumerate}[label=\textbf{Step \Roman*:},leftmargin=0.5cm]
	\item Choose a basis s.t. the operator $A$ looks like 
		\[ 
			\begin{pmatrix} 
				\lambda & * \\
				O & A'
			\end{pmatrix}
		\]
	\item Suppose the zero solution is stable, contain the phase curve to sit inside a small nbh. of the origin, in order to control $\n{f(x)}$
	\item use the estimation \[ \partial_t x_1^2 = 2\lambda x_1^2 + 2x_1f_1 \geq \lambda x_1^2, \] and Gronwall inequality to blows \[ x_1^2(t) \geq x_1^2(0) \cdot e^{\lambda t} \] up, arriving at the contradiction.
	\end{enumerate}
\end{enumerate}
\end{sketchproof}
\underline{Problem 2(4)}
The linear part of the field has $\det A =3; \text{tr} A=-3$ hence both of the eigenvalues are negative, while the nonlinear part 
\[ 
	-(x_1^2+x_2^2)\begin{pmatrix} 
	x_1 \\
	x_2
	\end{pmatrix}=O(\n{x}^3)
\]
hence the zero solution of the system is stable.
\subsection{Non-Autonomous Systems}
\cite{2001}
\subsection{Constructions of Liapunov V function}

\paragraph{If the dynamical system is stable, then }
\paragraph{If it is a planar vector field, and $x_2$ is the velocity of $x_1$}
\paragraph{If it is a constant coeffient linear vector field,}
We choose such $V$ from the set of quadratic forms ${x^T B x}$, which is essentially the set of real symmetric matrices $\mathscr{S}$, then the derivative of $V$ along the flow of the vector field ${Ax}$ is just $x^T (A^T B+ BA)x=: x^TCx$. 

As we know, in the topological vector space of all real matrices of size $n\times n$ $M_n(\mathbb{R})$ \[ (d_{ij})\in \mathscr{S} \iff (d_{ij}) \in \cap_{ij} \ker(d_{ij} - d_{ji}) =
\begin{bmatrix}
	d_{12} - d_{21} \\
	\vdots \\
	d_{n-1,n} - d_{n,n-1}
\end{bmatrix} 
^{-1}
(0).
\]
In other words, $\mathscr{S} \subset M_n(\mathbb{R})$ is a topological linear closed subspace.

Looking at this linear (hence continuous) operator $A^T * + * A$ on the space of real symmetric matrices more closely, 

\begin{lemma}\label{lemma:isolinear} if eigenvalues of $A$ satisfy \[ \lambda_i + \lambda_j \neq 0, \]
then the operator $A^T * + * A$ is injective hence bijective.
\end{lemma}
Assuming $C=(c_{ij}), A=(a_{ij})$

For planar dynamical system, 

\[ V(x)
=
\frac{-1}{\Delta}
\begin{vmatrix}
	0 &	x_1^2 & 2 x_1x_2 & x_2^2 \\
	c_{11} & a_{11} & a_{21} & 0 \\
	2c_{12} & a_{12} & a_{11} + a_{22} & a_{21} \\
	c_{22} & 0 & a_{12} & a_{22}
\end{vmatrix}
\]
where $\Delta = tr A \times \det A \neq 0.$

\bibliography{DynSys_ref}
\bibliographystyle{plain}

\end{document}
